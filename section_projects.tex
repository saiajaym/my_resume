% Awesome Source CV LaTeX Template
%
% This template has been downloaded from:
% https://github.com/darwiin/awesome-neue-latex-cv
%
% Author:
% Christophe Roger
%
% Template license:
% CC BY-SA 4.0 (https://creativecommons.org/licenses/by-sa/4.0/)

%Section: Project
\vspace{-2em}
\sectionTitle{Academic Projects}{\faLaptop}

% \begin{projects}
% 	\projectnew
% 	{Board Games}
% 	{Guide: Prof. Deepak Khemani
% 	\begin{itemize}
% 	\item \textbf{Othello Board}: Developed a best move bot for Othello game using AlphaBeta pruning algorithm given a particular configuration of the coins.
% 	\item \textbf{8-Puzzle Board}: Developed a Goal-State search approach to solve 8-ouzzle and 15-puzzle boards using A* search algorithm.
% 	\end{itemize}
% 	}
% 	{Artificial Intelligence}
	
%     \projectnew
% 	{Hearst Patterns for mining Ontology from text}
% 	{Guide: Prof. Suthanu Chakraborty \newline
% 	Developed a bootstrapper to enhance an existing Ontology (type of) relations from large text data by repetitive matching of Hearst Patterns in text with existing Ontology and include feasible relations to the set in every recursion.
% 	}
% 	{Natural Language Processing}

% \end{projects}
%



%     % \experience
%     % {Present}     {Graduate Research Assistant}{Advisor: Prof. Anna Squicciarini}{Penn State University \hfill Jan 2020 - Present}
%     % {Sep 2019}    {
%     %                 \textbf{Goal-Oriented Information Extraction in Spoken Dialogue Systems}
%     %                 \begin{itemize}
%     %                     \item Developed a novel method to extract \textbf{frame-semantics} related to information disclosure in user-generated online text. Submitted our work at \textbf{EMNLP 2020}*.
%     %                     \item Implemented \textbf{semi-supervised adversarial} models for multi-label text classification tasks.
%     %                 \end{itemize}
%     %                 }
%     %                 {Natural Language Processing, Deep Learning, Information Extraction}

%     % %\emptySeparator
%     % \vspace{-0.5em}
    
    
%     \experience
%     {Present}     {Object Tracking with Capsule Networks,}{Guide: Prof. Robert Collins}{Penn State University \hfill Jan 2020 - Apr 2020}
%     {Sep 2019}    {
%                       \begin{itemize}
%                         \item Studied the effect of \href{https://arxiv.org/pdf/1710.09829.pdf}{\color{accentcolor}{capsule networks}} on \textbf{object tracking} of different motion classes using the \href{http://got-10k.aitestunion.com/}{\color{accentcolor}{GOT-10K}} dataset.
%                         \item Modified \href{https://www.robots.ox.ac.uk/~luca/siamese-fc.html}{\color{accentcolor}{SiamFC}} object tracker with additional layers of capsule networks. 
%                         [\href{https://github.com/chandan047/RAT-Tracker}{\color{accentcolor}{code}}]
%                         \item Observed \textbf{improvements} in Illumination Variation, Background Clutters, Low Resolution and Motion Blur classes.
%                       \end{itemize}
%                     }
%                     {Computer Vision, Deep Learning, Object Tracking, Capsule Networks}
                    
%   \vspace{-0.5em}
    
    
    
    
%     \experience
%     {Present}     {Robust Semantic Role Labeling,}{Guide: Prof. Shomir Wilson}{Penn State University \hfill Jan 2020 - Apr 2020}
%     {Sep 2019}    {
%                       \begin{itemize}
%                         \item Trained a \href{https://www.aclweb.org/anthology/D19-1432.pdf}{distributionally robust} model for \href{https://www.aclweb.org/anthology/P00-1065.pdf}{Semantic Role Labeling} task in Natural Language Processing. [\href{https://github.com/chandan047/Robust-SRL}{\color{accentcolor}{link}}]
% 	                    \item Obtained \textbf{better} performance on \textbf{low-represented domains} while keeping the overall performance same. 
%                       \end{itemize}
%                     }
%                     {Natural Language Processing, Deep Learning, Robust Model}
                    
%   \vspace{-0.5em}
    

%     \experience
%     {Present}     {Sequence Image Captioning,}{Guide: Prof. C. Lee Giles}{Penn State University \hfill Sep 2019 - Nov 2019}
%     {Sep 2019}    {
%                       \begin{itemize}
%                         \item Implemented and trained sequence image captioning using \textbf{hybrid network} of \href{https://arxiv.org/pdf/1805.10973.pdf}{\color{accentcolor}{GLAC}} and \href{https://arxiv.org/pdf/1909.02701.pdf}{\color{accentcolor}{VSRN}}.
%                         \item Applied \textbf{Graph Convolutional Networks} on image sequences in story to visualize affinity between regions. 
%                 	   % \item Extension to prior research on sequence image captioning using \href{http://visionandlanguage.net/VIST/}{VIST} dataset from Microsoft Research
%                       \end{itemize}
%                     }
%                     {Computer Vision, Natural Language Processing, Deep Learning, Vision-to-Language}
                    
% %   \emptySeparator
  
% %   \newpage
  
% %   \experience
% %     {Present}     {Course Project}{Artificial Intelligence}{IIT Madras}
% %     {Sep 2019}    {
% %                     \textbf{Project: Adversarial Board Games\newline Guide: Dr. Deepak Khemani}
% %                       \begin{itemize}
% %                         \item \textbf{Othello}: Implemented a game bot for Othello using adversarial algorithms to determine the next best move. This bot stood second among all bots from the course.
% %                         \item \textbf{15-Puzzle}: Implemented Goal-State search algorithms to solve 8-puzzle and 15-puzzle boards.
% %                       \end{itemize}
% %                     }
% %                     {Game Theory, Artificial Intelligence}
% %   \emptySeparator
  
% \end{experiences}


% \newcommand\educat[6]{
%   \textbf{#1}    & \textbf{#2, \textsc{#3}, #4}   								\\*
%   \textbf{#5}    & \begin{minipage}[t]{\rightcolumnlength}
%   					         #6\
%                   \end{minipage}										\\*
                 
%   }

%RAT, roubust SRL, Zebra, 


\begin{projects}
\vspace{0.2em}
\projectentry{\gitchandan{FewShot NER}{MetaLearningForNER}}
   {
    \begin{myitemize}
        \item Achieved state of the art results for Named Entity Recognition in FewShot learning setting.
        \item  Work to be submitted at ACL \href{https://2021.aclweb.org/calls/papers/}{\color{accentcolor}{ACL-IJCNLP 2021}}.
    \end{myitemize}
    }

   \projectentry{\gitchandan{Robust Semantic Role Labeling (SRL)}{Robust-SRL}}
   {
    \begin{myitemize}
        \item Developed a \href{https://www.aclweb.org/anthology/D19-1432.pdf}{\color{accentcolor}{distributionally robust}} model for underrepresented classes in  \href{https://www.aclweb.org/anthology/P00-1065.pdf}{\color{accentcolor}{Semantic Role Labeling}}.
        \item The model achieved \textbf{4$\%$ absolute F1 increase} on domains with low training data and \textbf{1.1\% absolute F1 increase} overall.
    \end{myitemize}
    }
\vspace{0.3em}



\projectentry{\gitchandan{Object Tracking Using CapsuleNets}{RAT-Tracker}}
    {\begin{myitemize}
        \item Demonstrated a novel method of using \href{https://arxiv.org/pdf/1710.09829.pdf}{\color{accentcolor}{capsule networks}} for object tracking with \href{https://www.robots.ox.ac.uk/~luca/siamese-fc.html}{\color{accentcolor}{SiamFC}} as baseline and \href{http://got-10k.aitestunion.com/}{\color{accentcolor}{GOT-10K}} dataset.
        \item Achieved improvements over \href{https://www.robots.ox.ac.uk/~luca/siamese-fc.html}{\color{accentcolor}{baseline}} on classes like Illumination Var.($+2\%$), Background Clutter($+1.8\%$), Low Res.($+0.8\%$). 
    \end{myitemize}}

%\vspace{0.5em}
    
    \projectentry{ \githubproject{Zebra identification using Deep Learning}{ZebraRecognition}}
    {
\begin{myitemize}
    \item A PoC: uses \textbf{MaskRCNN} to identify various stripes on zebra, and encode stripe information to uniquely identify a Zebra.
    
\end{myitemize}
    }

%\vspace{0.3em}
   \projectentry{ \githubproject{Torrent like file sharing using GO language}{gorrent}}
    {\begin{myitemize}
        \item An attempt of replicating ideas from \href{http://bittorrent.org/bittorrentecon.pdf}{\color{accentcolor}{BitTorrent paper}} with built in concurrenncy and fault tolerence.
    \end{myitemize}}
    
\end{projects}